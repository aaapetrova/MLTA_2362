\problemset{Математическая логика и теория алгоритмов}
\problemset{Индивидуальное домашнее задание №2}	% поменяйте номер ИДЗ

\renewcommand*{\proofname}{Решение}
\newcommand*\circled[1]{\tikz[baseline=(char.base)]{\node[shape=circle,draw,inner sep=2pt] (char) {#1};}}
\newcommand*\ccircled[1]{\tikz[baseline=(char.base)]
{\node[shape=circle,draw=black, fill=gray, opacity=0.8,inner sep=2pt] (char) {#1};}}
\newcommand*\bicircled[1]{\tikz[baseline=(char.base)]{\node[shape=circle,draw,inner sep=1pt] (char) {#1};}}
\shorthandoff{"}

%%%%%%%%%%%%%% ЗАДАНИЕ №1 %%%%%%%%%%%%%%
%% Условие задания №1
\begin{problem} 
Найдите резольвенты первого порядка следующего набора дизъюнктов: \\$B \lor \bar C \lor \bar D \lor \bar E, A \lor B \lor \bar C \lor \bar D \lor \bar E, \bar A \lor \bar D, \bar A \lor B \lor E$
\end{problem}

%% Решение задания №1
\begin{proof} $ $\\
    $R_1 = B \lor \bar C \lor \bar D \lor \bar E$\\
    $R_2 = A \lor B \lor \bar C \lor \bar D \lor \bar E$\\
    $R_3 = \bar A \lor \bar D$\\
    $R_4 = \bar A \lor B \lor E$\\

    \begin{enumerate}
      \item $R_1,R_4:$ $\bar A \lor B \lor \bar C \lor \bar D$
      \item $R_2,R_3:$ $B \lor \bar C \lor \bar D \lor \bar E$
      \item $R_2,R_4:$ $1$
    \end{enumerate}
\end{proof}

%%%%%%%%%%%%%% ЗАДАНИЕ №2 %%%%%%%%%%%%%%
%% Условие задания №2
\begin{problem}
    Приведите данную формулу к ПНФ: \\$\lnot (\exists y R(b, y) \oplus \lnot \forall x P(x, c, x))$
\end{problem}

%% Решение задания №2
\begin{proof} $ $\\
    $\lnot (\exists y R(b, y) \oplus \lnot \forall x P(x, c, x)) = \lnot ((\exists y R(b,y) \land \forall x P(x, c, x)) \lor (\lnot \exists y R(b,y) \land \lnot \forall x P(x, c, x))) = (\lnot \exists y R(b,y) \lor \lnot x P(x, c, x)) \land (\exists y R(b, y) \lor \forall x P(x,c,x)) = \forall y \exists x (\lnot R(b,y) \lor \lnot P(x,c,x)) \land \exists y \forall x (R(b,y) \lor P(x,c,x)) = \forall x \exists y \forall z \exists w (R(b,y) \lor P(x,c,x))(\lnot R(b,z) \lor \lnot P(w,c,w))$
\end{proof}

%%%%%%%%%%%%%% ЗАДАНИЕ №3 %%%%%%%%%%%%%%
%% Условие задания №3
\begin{problem}
    Приведите данную формулу к СНФ: \\$\forall x \exists y \forall z \exists t \forall u ((R(t) \land Q(x, y, z)) \land P(u))$
\end{problem}

%% Решение задания №3
\begin{proof} $ $\\
    $\forall x \exists y \forall z \exists t \forall u ((R(t) \land Q(x, y, z)) \land P(u)) = \forall x \forall z \exists t \forall u ((R(t) \land Q(x, f(x), z)) \land P(u)) = \forall x \forall z \forall u ((R(g(x,z)) \land Q(x, f(x), z)) \land P(u))$\\
    $y = f(x)$\\
    $t = g(x,z)$
\end{proof}

%%%%%%%%%%%%%% ЗАДАНИЕ №4 %%%%%%%%%%%%%%
%% Условие задания №4
\begin{problem}
    Унифицируйте данные формулы:\\
    $ S(q(h(q(r(x)), \varphi(a))), q(h(r(x), q(p(b)))), h(x, p(b)), f(q(\varphi(a)), u), s(r(x))) \\  S(q(h(q(z), y)), q(h(z, q(t))), h(c, t), f(q(y), g(t)), s(z))$
\end{problem}

%% Решение задания №4
\begin{proof} $ $\\
   $c=x$\\
   $z=r(x)$\\
   $y=\varphi(a)$\\
   $t=p(b)$\\
   $u=g(t)$\\\\
   $ S(q(h(q(r(x)), \varphi(a))), q(h(r(x), q(p(b)))), h(x, p(b)), f(q(\varphi(a)), g(t)), s(r(x)))$
\end{proof}

%%%%%%%%%%%%%% ЗАДАНИЕ №5 %%%%%%%%%%%%%%
%% Условие задания №5
\begin{problem}
     Придумайте интерпретацию, для которой данная формула а) верна; б) неверна; или докажите, что это невозможно. $\neg((\forall x P(a, x) \wedge S(c, c, c)) \Rightarrow \forall y Q(y))$
\end{problem}

%% Решение задания №5
\begin{proof} $ $\\
    $ \forall x P(a, x) \wedge S(c, c, c) \land \exists y \neg Q(y)$\\
    $\sqsupse M = [0; 1]$ \\
    а) Верна:\\
    S(c, c, c) - "$c = c = c$" \\
    P(a, x) - "$x + a \geq 0$" \\
    Q(y) - "$y < 0$" \\\\
    б) Неверна:\\
    S(c, c, c) - "$c = c = c$" \\
    P(a, x) - "$x + a \geq 0$" \\
    Q(y) - "$y \geq 0$" \\
\end{proof}

%%%%%%%%%%%%%% ЗАДАНИЕ №6 %%%%%%%%%%%%%%
%% Условие задания №6
\begin{problem}
    а) Опишите язык, заданный данной грамматикой. б) Удовлетворяет ли он условию однозначности ветвления?\\
    $A::=z \mid d A d$
\end{problem}

%% Решение задания №6
\begin{proof} $ $\\\\
    а) $d^n z d^n, n \geq 0$ \\
    б) Удовлетворяет: \\
    \begin{enumerate}
        \item Правило A::=z позволяет сразу закончить построение, так как z — конечный символ.
        \item Правило A::=dAd однозначно показывает, что к базовой строке A добавляются символы d с двух сторон.
    \end{enumerate}
\end{proof}

%%%%%%%%%%%%%% ЗАДАНИЕ №7 %%%%%%%%%%%%%%
%% Условие задания №7
\begin{problem}
     Дана грамматика некоторого языка:\\
    $\begin{aligned} & S::=C \\ & B::=S \\ & C::=a C \mid a D \\ & D::=a S|b D| b B \mid \Lambda\end{aligned}$\\
    Постройте (любой) конечный автомат, распознающий этот язык.
\end{problem}

%% Решение задания №7
\begin{proof} $ $\\
    \begin{center}
        \begin{tikzcd}[row sep=4em, column sep=4em]
        \ccircled{S} \arrow[d, "\Lambda" description] &
        \circled{B} \arrow[l, "\Lambda" description]
        \\
        \circled{C} \arrow[r, "a" description] 
        \arrow[loop, in= 180, out = 270, looseness = 4,  "a" description] &
        \bicircled{\circled{D}} \arrow[lu, "a" description] 
        \arrow[loop, in= 360, out = 270, looseness = 4,  "b" description]
        \arrow[u, "b" description] 
        \end{tikzcd}
    \end{center}
\end{proof}

%%%%%%%%%%%%%% ЗАДАНИЕ №8 %%%%%%%%%%%%%%
%% Условие задания №8
\begin{problem}
     Постройте детерминированный конечный автомат, эквивалентный данному:\\
     \begin{center}
\begin{tikzcd}[row sep=4em, column sep=4em]
\ccircled{q0} \ar[r, bend left = 10, rightarrow, "\epsilon" description] \ar[loop, in = 120, out = 180, looseness = 4, "a" description] \ar[dr, bend left = 10, rightarrow, "c" description] \ar[d, bend left = 10, rightarrow, "a" description] &
\circled{q2} \ar[l, bend left = 10, rightarrow, "c" description] \ar[ld, bend left = 10, rightarrow, "c" description]
\ar[l, bend left = 27, rightarrow, "\epsilon" description]
\\
\circled{q1} \ar[u, bend left = 10, rightarrow, "c" description] \ar[loop, out = 270, in = 180, looseness = 4, "b" description] &
\bicircled{\circled{q3}} \ar[loop, out = 0, in = 300, looseness = 4, "b" description] \ar[ul, bend left = 10, rightarrow, "a" description]
\end{tikzcd}
\end{center}
\end{problem}

%% Решение задания №8
\begin{proof} $ $ \\\\
Пустые переходы:\\
\begin{tikzcd}[row sep=4em, column sep=4em]
\circled{q0} \ar[r, bend left = 10, rightarrow, "\epsilon" description] & 
\circled{q2} \ar[l, bend left = 10, rightarrow, "\epsilon" description]
\end{tikzcd}\\

Итоговый детерминированный конечный автомат:\\
\begin{center}
\begin{tikzcd}[row sep=4em, column sep=4em]
\circled{q0q2} \ar[r, bend left = 10, rightarrow, "c" description]
            \ar[dr, bend left = -10, rightarrow, "a" description]&
\bicircled{\circled{q3}} \ar[loop, out = 15, in = 90, looseness = 4, "b" description]
            \ar[l, bend left = 10, rightarrow, "a" description] \\
\circled{q1} \ar[loop, out = 195, in = 270, looseness = 4, "b" description] 
            \ar[u, bend left = 10, rightarrow, "c" description] &
\circled{q0q1} \ar[loop, out = 220, in = 300, looseness = 4, "a" description]
            \ar[l, bend left = 10, rightarrow, "b" description]
            \ar[r, bend left = 10, rightarrow, "c" description] &
\bicircled{\circled{q0q3}} \ar[l, bend left = 10, rightarrow, "a" description]
            \ar[ul, bend left = 10, rightarrow, "c" description]
            \ar[ul, bend left = -10, rightarrow, "b" description]
\end{tikzcd}
\end{center}
\end{proof}

%%%%%%%%%%%%%% ЗАДАНИЕ №9 %%%%%%%%%%%%%%
%% Условие задания №9
\begin{problem}
    Постройте регулярное выражение, задающее язык, распознаваемый этим автоматом.
    \begin{center}
    \begin{tikzcd}[row sep=4em, column sep=4em]
    \ccircled{q0} \ar[loop, out = 165, in = 100, looseness = 4, "c" description]
                \ar[r, bend left = 10, rightarrow, "a" description]
                \ar[d, bend left = 10, rightarrow, "a" description] &
    \bicircled{\circled{q2}} \ar[loop, out = 90, in = 15, looseness = 4, "b" description]
                \ar[ld, bend left = 10, rightarrow, "a" description] \\
    \circled{q1} \ar[loop, out = 270, in = 195, looseness = 4, "b" description] 
                \ar[ur, bend left = 10, rightarrow, "b" description] &
    \end{tikzcd}
    \end{center}
\end{problem}

%% Решение задания №9
\begin{proof} $ $\\\\
     \begin{cases} 
     q_0 = cq_0 + aq_2 + aq_1 \\
     q_1 = bq_1 + bq_2 \\
     q_2 = bq_2 + aq_1 + \varepsilon
     \end{cases} \\\\
     \begin{cases} 
     q_0 = c^* a (q_2 + q_1) \\
     q_1 = b^* b q_2 \\
     q_2 = b^* (a q_1 + \varepsilon) = b^*aq_1 + b^*
     \end{cases} \\\\
     $q_1=b^*b(b^*aq_1+b^*)=b^*bb^*aq_1+b^*bb^*=(b^*bb^*a)^*b^*bb^*$\\
     $q_2=b^*ab^*bq_2+b^*=(b^*ab^*b)^*b^*$\\
     $q_0=c^*a((b^*bb^*a)^*b^*bb^*+(b^*ab^*b)^*b^*)$
\end{proof}

%%%%%%%%%%%%%% ЗАДАНИЕ №10 %%%%%%%%%%%%%%
%% Условие задания №10
\begin{problem}
    Постройте автомат, распознающий язык, задаваемый
этим регулярным выражением: $(a(b + (cd)^*)e)^*$
\end{problem}

%% Решение задания №10
\begin{proof} $ $\\
    \begin{center}
    \begin{tikzcd}[row sep=4em, column sep=4em]
    \bicircled{\circled{S}}  \ar[r, bend left = 10, rightarrow, "a" description] \ar[rd, bend left = 10, rightarrow, "a" description] &
    \circled{q1} \ar[loop, out = 90, in = 20, looseness = 4, "cd" description] \ar[dl, bend left = 10, rightarrow, "\epsilon" description] \\
    \circled{q2} \ar[u, bend left = 10, rightarrow, "e" description] & 
    \circled{q3} \ar[l, bend left = 10, rightarrow, "b" description]
    \end{tikzcd}
    \end{center}
\end{proof}

%%%%%%%%%%%%%% ЗАДАНИЕ №11 %%%%%%%%%%%%%%
%% Условие задания №11
\begin{problem}
    Постройте детерминированный конечный автомат, распознающий слова в алфавите {i, l, z}, которые не оканчиваются на zzi.
\end{problem}

%% Решение задания №11
\begin{proof} $ $\\
    \begin{center}
    \begin{tikzcd}[row sep=4em, column sep=4em]
    \bicircled{\circled{S}} \ar[loop, out = 90, in = 130, looseness = 3, "i" description]
                            \ar[loop, out = 150, in = 190, looseness = 3, "l" description] 
                            \ar[r, bend left = 10, rightarrow, "z" description] &
    \bicircled{\circled{q1}} \ar[l, bend left = 5, rightarrow, "i" description]
                             \ar[l, bend left = 20, rightarrow, "l" description]
                             \ar[ld, bend left = 10, rightarrow, "z" description] \\
    \bicircled{\circled{q2}} \ar[u, bend left = 20, rightarrow, "l" description]
                             \ar[loop, out = 270, in = 210, looseness = 3, "z" description] 
                             \ar[r, bend left = -10, rightarrow, "i" description] & 
    \circled{q3} \ar[u, bend left = -10, rightarrow, "z" description]
                 \ar[lu, bend left = -10, rightarrow, "i" description]
                 \ar[lu, bend left = 10, rightarrow, "l" description]
    \end{tikzcd}
    \end{center}
\end{proof}

%%%%%%%%%%%%%% ЗАДАНИЕ №12 %%%%%%%%%%%%%%
%% Условие задания №12
\begin{problem}
    Дана машина Тьюринга с начальным состоянием q0 и конечным состоянием $q_2$. Какой результат даст эта машина Тьюринга для ленты kdddk? Считывающая головка находится на крайнем левом символе. Пустой символ — *. \\
    $q_1$*$ \rightarrow q_0 dR \\
     q_1 d \rightarrow q_1 kL \\
     q_1 k \rightarrow q_0 kL \\
     q_0 d \rightarrow q_2 kL \\ 
     q_0 k \rightarrow q_1 dR$
\end{problem}

%% Решение задания №12
\begin{proof} $ $\\
    \begin{enumerate}
        \item $\overset{q_0}{k}dddk$ 
        \item $d\overset{q_1}{d}ddk$
        \item $\overset{q_1}{d}kddk$
        \item $\overset{q_1}{*}kkddk$
        \item $d\overset{q_0}{k}kddk$
        \item $dd\overset{q_1}{k}ddk$
        \item $d\overset{q_0}{d}kddk$
        \item $\overset{q_2}{d}kkddk$
    \end{enumerate}
\end{proof}

%%%%%%%%%%%%%% ЗАДАНИЕ №13 %%%%%%%%%%%%%%
%% Условие задания №13
\begin{problem}
    а) Что следующий алгоритм Маркова делает со словом zxyxyx? \\
    б) Из некоторого слова после применения 1 шага данного алгоритма Маркова получилось yzxxyzy. Каким могло быть исходное слово? \\
    $zzz \rightarrow zx \\
     yzx \rightarrow zx \\
     xxx \rightarrow zy \\
     yxx \rightarrow zy \\ 
     xy \rightarrow zxx$
\end{problem}

%% Решение задания №13
\begin{proof} $ $\\
    а) \begin{enumerate}
        \item z\textbf{xy}xyx
        \item zz\textbf{xxx}yx
        \item\textbf{zzz}yyx 
        \item z\textbf{xy}yx
        \item zzx\textbf{xy}x
        \item zzxz\textbf{xxx}
        \item zzxzzy
    \end{enumerate} \\
    б) \\
    y\textbf{zx}xyzy \leftarrow y\textbf{zzz}xyzy \\
    y\textbf{zx}xyzy \leftarrow y\textbf{yzx}xyzy \\
    y\textbf{zxx}yzy \leftarrow y\textbf{xy}yzy \\
    
\end{proof}

%%%%%%%%%%%%%% ЗАДАНИЕ №14 %%%%%%%%%%%%%%
%% Условие задания №14
\begin{problem}
    Дан автомат с магазинной памятью. Входной алфавит {a, x}; алфавит стека {p}; q0 — начальное состояние, q1 — конечное.\\
    Правила\\
    1) $q_0 \epsilon a \rightarrow q_1 p$  2) $q_0 \epsilon x \rightarrow q_0 p$  3) $q_0 pa \rightarrow q_0 p$\\
    4) $q_0 px \rightarrow q_0 \epsilon$  5) $q_1 \epsilon a \rightarrow q_0 \epsilon$  6) $q_1 \epsilon x \rightarrow q_0 \epsilon$\\
    7) $q_1 pa \rightarrow q_1 p$  8) $q_1 px \rightarrow q_0 p$\\
    Придумайте пример шестибуквенного слова в алфавите {a, x}, которое этот автомат а) распознаёт; б) не распознаёт. в) Есть ли у этого автомата бесполезные правила, которые не будут выполняться ни при каком поданном на вход автомата слове?

\end{problem}

%% Решение задания №14
\begin{proof} $ $\\
    \begin{center}
    \begin{tikzcd}[row sep=4em, column sep=4em]
    & \circled{$q_0 \epsilon$}  \ar[ld, bend left = -10, rightarrow, "a" description] \ar[rd, bend left = 15, rightarrow, "x" description] & \\
    \circled{$q_1 p$} \ar[loop, out = 220, in = 160, looseness = 4, "a" description] \ar[rr, bend left = -10, rightarrow, "x" description] & &
    \circled{$q_0 p$} \ar[loop, out = 300, in = 10, looseness = 4, "a" description] \ar[ul, bend left = 10, rightarrow, "x" description]
    \end{tikzcd}
    \end{center}\\
    а) не существует \\
    б) $axaxax$ \\
    в) 5) и 6)
\end{proof}
