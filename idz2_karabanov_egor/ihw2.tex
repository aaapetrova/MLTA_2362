\problemset{Комбинаторика и теория графов}
\problemset{Индивидуальное домашнее задание №2}	% поменяйте номер ИДЗ

\renewcommand*{\proofname}{Решение}
%%%%%%%%%%%%%% ЗАДАНИЕ №1 %%%%%%%%%%%%%%
%% Условие задания №1
\begin{problem}
	Определить, является ли данный граф эйлеровым, полуэйлеровым, гамильтоновым, полугамильтоновым, двудольным, вершинно-двусвязным, рёберно-двусвянзным. Построить дерево блоков и точек сочленения.
    \begin{center}
    \begin{tikzcd}
		& A  \ar[r, dash]  \ar[d, dash] 
		& B  
		\\
		C   \ar[dr, dash] \ar[d, dash]  
		& D \ar[dl, dash] \ar[r, dash] \ar[d, dash]   
		& E \ar[d, dash] \ar[r, dash]  
		& F \ar[d, dash] 
		\\
		G   \ar[ur, dash]   
		& H \ar[u, dash] 
		& I \ar[l, dash] \ar[r, dash]
		& J \ar[ul, dash]
        \\
        & K \ar[ur, dash] \ar[r, dash]
        & L \ar[ul, dash]
	\end{tikzcd}\\
    \end{center}
\end{problem}

%% Решение 1
%% Решение задания №1
\pagebreak
\begin{proof} $ $\\
    \begin{itemize}
        \item Не является эейлеровым, т.к. не все вершины имеют четную степень
        \item Является полуэйлеровым, т.к. ровно 2 вершины имеют нечентную степень: B, J
        \item Не является гамильтоновым
        \item Является полугамильтоновым, т.к. существует путь, проходящий по всем вершинам ровно 1 раз: B, A, D, E, F, J, I, K, L, H, G, C
        \item Не является двудольным, т.к. смежные вершины I, J, F оказываются одного цвета при раскрашивании графа
        \begin{center}
        \begin{tikzcd}
		& \textcolor{red}{A}  \ar[r, dash]  \ar[d, dash] 
		& \textcolor{green}{B}  
		\\
		\textcolor{green}{C}   \ar[dr, dash] \ar[d, dash]  
		& \textcolor{green}{D} \ar[dl, dash] \ar[r, dash] \ar[d, dash]   
		& \textcolor{red}{E} \ar[d, dash] \ar[r, dash]  
		& \textcolor{green}{F} \ar[d, dash] 
		\\
		\textcolor{red}{G}   \ar[ur, dash]   
		& \textcolor{red}{H} \ar[u, dash] 
		& \textcolor{green}{I} \ar[l, dash] \ar[r, dash]
		& \textcolor{green}{J} \ar[ul, dash]
        \\
        & \textcolor{red}{K} \ar[ur, dash] \ar[r, dash]
        & \textcolor{green}{L} \ar[ul, dash]
	    \end{tikzcd}
        \end{center}
        \item Не является вершинно двусвязным, т.к. в графе присутствуют шарниры: A, D
        \item Не является реберно двусвязным, т.к. в графе присутствуют мосты: BA, AD
        \item Дерево блоков и точек сочленения:\\
        \tikz {
        \path
        (0, 0) node[white, circle,fill=black] (AB) {AB}
        (2, 0) node[circle,fill=red] (A) {A}
        (4, 0) node[white, circle,fill=black] (AD) {AD}
        (6, 0) node[circle,fill=red] (D) {D}
        (8, 0) node[white, circle, fill=black] (other) {DGCHLKIJFE};

        \draw (AB) -- (A);
        \draw (A) -- (AD);
        \draw (AD) -- (D);
        \draw (D) -- (other);
        }

    \end{itemize}
\end{proof}

\begin{problem}
\end{problem}
\begin{proof}
\end{proof}

\begin{problem}
\end{problem}
\begin{proof}
\end{proof}

\vspace{10mm}
%% Условие задания №4
\begin{problem} $ $\\
	\begin{itemize}
	    \item[a)] построить код Прюфера для данного дерева:
        \vspace{4mm}
        \begin{center}
        \begin{tikzcd}
        1 \ar[d, dash] \ar[r, dash]
        &2
        &3 \ar[dll, dash]
        &4 \ar[r, dash]
        &5 \ar[r, dash] \ar[dllll, dash] \ar[dlll, dash]
        &6 \ar[dl, dash]
        \\
        7 
        &8 \ar[r, dash]
        &9 \ar[r, dash]
        &10
        &11
        \end{tikzcd}
        \end{center}
        \vspace{4mm}
        \item [б)] Построить дерево по коду Прюфера: 2 11 2 3 3 4 5 9 5
	\end{itemize}
\end{problem}

%% Решение задания 4
\begin{proof} $ $\\
    \begin{itemize}
        \item[a)] Код Прюфера: 1 7 7 5 5 9 8 5 6
        \item[б)] Получившиеся дерево по коду Прюфера:
        \begin{center}
        \begin{tikzcd}
        1 \ar[r, dash]
        &2 \ar[dl, dash] \ar[r, dash]
        &3 \ar[dl, dash] \ar[r, dash]
        &4 \ar[r, dash]
        &5 \ar[d, dash] \ar[dll, dash]
        &6 \ar[dl, dash]
        \\
        7 
        &8 
        &9 \ar[r, dash]
        &10
        &11
        \end{tikzcd}
        \end{center}
    \end{itemize}
\end{proof}

%% Условие задания №5
\begin{problem} При помощи плгоритма Kosaraju найти компоненты сильной связности данного графа:\\
    \begin{center}
        \begin{tikzcd}
            A \ar[r, rightarrow]
            &B \ar[d, rightarrow] \ar[ddl, leftrightarrow]
            &C \ar[r, leftrightarrow]
            &D \ar[llldd, rightarrow]
            \\
            E 
            &F \ar[urr, rightarrow]
            &G \ar[ur, rightarrow] \ar[r, rightarrow] \ar[ddr, leftrightarrow]
            &H 
            \\
            I \ar[r, leftarrow]
            &J \ar[ld, leftrightarrow]
            &K \ar[dl,  rightarrow]
            &L \ar[d, leftrightarrow]
            \\ 
            M \ar[u, rightarrow]
            &N \ar[l, leftrightarrow]
            &O \ar[u, rightarrow]
            &P \ar[l, rightarrow]
        \end{tikzcd}
    \end{center}
\end{problem}

%% Решение задания 5
\begin{proof} $ $\\
    Начинаем поиски в глубину с вершин: A, E, L\\
    Полученный стек: [I, C, D, F, B, A, E, H, G, J, M, N, K, O, P, L\\
    Транспонированный граф:
        \begin{center}
        \begin{tikzcd}
            A \ar[r, leftarrow]
            &B \ar[d, leftarrow] \ar[ddl, leftrightarrow]
            &C \ar[r, leftrightarrow]
            &D \ar[llldd, leftarrow]
            \\
            E 
            &F \ar[urr, leftarrow]
            &G \ar[ur, leftarrow] \ar[r, leftarrow] \ar[ddr, leftrightarrow]
            &H 
            \\
            I \ar[r, rightarrow]
            &J \ar[ld, leftrightarrow]
            &K \ar[dl,  leftarrow]
            &L \ar[d, leftrightarrow]
            \\ 
            M \ar[u, leftarrow]
            &N \ar[l, leftrightarrow]
            &O \ar[u, leftarrow]
            &P \ar[l, leftarrow]
        \end{tikzcd}
        \end{center}
        \vspace{5mm}
    Граф после поисков в глубину в порядке доставания вершин из стека и окрашивания вершин в рамках одного поиска:
        \begin{center}
        \begin{tikzcd}
            \textcolor{pink}{A} \ar[r, leftarrow]
            &\textcolor{blue}{B} \ar[d, leftarrow] \ar[ddl, leftrightarrow]
            &\textcolor{blue}{C} \ar[r, leftrightarrow]
            &\textcolor{blue}{D} \ar[llldd, leftarrow]
            \\
            \textcolor{orange}{E} 
            &\textcolor{blue}{F} \ar[urr, leftarrow]
            &\textcolor{red}{G} \ar[ur, leftarrow] \ar[r, leftarrow] \ar[ddr, leftrightarrow]
            &\textcolor{brown}{H} 
            \\
            \textcolor{blue}{I} \ar[r, rightarrow]
            &\textcolor{green}{J} \ar[ld, leftrightarrow]
            &K \ar[dl,  leftarrow]
            &\textcolor{red}{L} \ar[d, leftrightarrow]
            \\ 
            \textcolor{green}{M} \ar[u, leftarrow]
            &\textcolor{green}{N} \ar[l, leftrightarrow]
            &\textcolor{yellow}{O} \ar[u, leftarrow]
            &\textcolor{red}{P} \ar[l, leftarrow]
        \end{tikzcd}
        \end{center}
        Таким образом, граф герца для данного графа:
        \begin{center}
        \begin{tikzcd}
                    E
        &LPG \ar[dl, rightarrow] \ar[r, rightarrow] \ar[drr, rightarrow]
        &H
        &
        \\
        O \ar[r, rightarrow]
        &K \ar[r, rightarrow]
        &NMJ \ar[r, rightarrow]
        &BFIDC
        \\
        &
        &A \ar[ur, rightarrow]
        \end{tikzcd}
        \end{center}
\end{proof}

\begin{problem}
\end{problem}
\begin{proof}
\end{proof}

\begin{problem}
\end{problem}
\begin{proof}
\end{proof}

\begin{problem}
Найдите наибольшее паросочетание в двудольном графе, заданном набором рёбер (a, $\gamma$) (a, $\delta$) (a, $\epsilon$) (b, $\beta$) (b, $\theta$) (c, $\theta$) (d, $\delta$) (d, $\zeta$) (d, $\theta$) (e, $\alpha$) (e, $\beta$) (e, $\eta$) (f, $\zeta$) (g, $\gamma$) (g, $\delta$) (h, $\gamma$)
\end{problem}

\begin{proof} $ $\\
Изначальный граф:
\begin{center}
\tikz {
        \path
(0, -7) node (S) {S}
(2, 0) node (a) {a}
(8, 0) node (1) {$\gamma$}
(2, -2) node (g) {g}
(8, -2) node (2) {$\delta$}
(2, -4) node (h) {h}
(8, -4) node (3) {$\varepsilon$}
(2, -6) node (b) {b}
(8, -6) node (4) {$\beta$}
(2, -8) node (c) {c}
(8, -8) node (5) {$\theta$}
(2, -10) node (d) {d}
(8, -10) node (6) {$\zeta$}
(2, -12) node (e) {e}
(8, -12) node (7) {$\alpha$}
(2, -14) node (f) {f}
(8, -14) node (8) {$\eta$}
(10, -7) node (F) {F};
\draw[->] (S) -- (a);
\draw[->] (S) -- (g);
\draw[->] (S) -- (h);
\draw[->] (S) -- (b);
\draw[->] (S) -- (c);
\draw[->] (S) -- (d);
\draw[->] (S) -- (e);
\draw[->] (S) -- (f);

\draw[<-] (F) -- (1);
\draw[<-] (F) -- (2);
\draw[<-] (F) -- (3);
\draw[<-] (F) -- (4);
\draw[<-] (F) -- (5);
\draw[<-] (F) -- (6);
\draw[<-] (F) -- (7);
\draw[<-] (F) -- (8);

\draw[->] (a) -- (1);
\draw[->] (a) -- (2);
\draw[->] (a) -- (3);
\draw[->] (g) -- (1);
\draw[->] (g) -- (2);
\draw[->] (h) -- (1);
\draw[->] (b) -- (4);
\draw[->] (b) -- (5);
\draw[->] (c) -- (5);
\draw[->] (d) -- (2);
\draw[->] (d) -- (5);
\draw[->] (d) -- (6);
\draw[->] (e) -- (4);
\draw[->] (e) -- (7);
\draw[->] (e) -- (8);
\draw[->] (f) -- (6);
}
\end{center} $ $\\
Ответ: (a, $\eta$) (b, $\beta$) (c, $\alpha$)
(d, $\varepsilon$) (e, $\theta$) (f, $\delta$)  (g, $\zeta$) (h, $\gamma$)

Пройденные пути:\\
\{S, a, $\gamma$, F\}, \{S, g, $\delta$, F\}, \{S, h, $\gamma$, a, $\varepsilon$, F\}, \{S, b, $\beta$, F\}, \{S, c, $\theta$, F\}, \{S, d, $\zeta$, F\}, \{S, e, $\alpha$, F\}

Граф после применения алгоритма:
\begin{center}
% \usetikzlibrary{graphs,automata,positioning}
\tikz {
        \path
(0, -7) node (S) {S}
(2, 0) node (a) {a}
(8, 0) node (1) {$\gamma$}
(2, -2) node (g) {g}
(8, -2) node (2) {$\delta$}
(2, -4) node (h) {h}
(8, -4) node (3) {$\varepsilon$}
(2, -6) node (b) {b}
(8, -6) node (4) {$\beta$}
(2, -8) node (c) {c}
(8, -8) node (5) {$\theta$}
(2, -10) node (d) {d}
(8, -10) node (6) {$\zeta$}
(2, -12) node (e) {e}
(8, -12) node (7) {$\alpha$}
(2, -14) node (f) {f}
(8, -14) node (8) {$\eta$}
(10, -7) node (F) {F};

\draw[<-] (S) -- (f);
\draw[<-] (F) -- (8);

\draw[->] (a) -- (1);
\draw[->] (a) -- (2);
\draw[<-, red] (a) -- (3);
\draw[->] (g) -- (1);
\draw[<-, red] (g) -- (2);
\draw[<-, red] (h) -- (1);
\draw[<-, red] (b) -- (4);
\draw[->] (b) -- (5);
\draw[<-, red] (c) -- (5);
\draw[->] (d) -- (2);
\draw[->] (d) -- (5);
\draw[<-, red] (d) -- (6);
\draw[->] (e) -- (4);
\draw[<-, red] (e) -- (7);
\draw[->] (e) -- (8);
\draw[->] (f) -- (6);
}
\end{center} $ $\\
Ответ: $\tilde{E} = \{a\varepsilon,  g\delta,  h\gamma,  b\beta,  c\theta,  d\zeta,  e\alpha\}$
\end{proof}