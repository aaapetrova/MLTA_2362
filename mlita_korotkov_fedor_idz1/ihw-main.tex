\documentclass{amsart}

\usepackage[utf8]{inputenc}
\usepackage[T2A]{fontenc}
\usepackage[english,russian]{babel}
\usepackage{amsthm,amsmath,amsfonts,amssymb}
\usepackage{fullpage}
\usepackage{eufrak}

%%% Дополнительная работа с математикой
\usepackage{amsfonts,amssymb,amsthm,mathtools} % AMS
\usepackage{amsmath}
\usepackage{icomma}

%% Шрифты
\usepackage{euscript}	 % Шрифт Евклид
\usepackage{mathrsfs} % Красивый матшрифт

%% Свои команды
\DeclareMathOperator{\sgn}{\mathop{sgn}}	% сигнум
\DeclareMathOperator{\cov}{\mathop{cov}}	% ковариация
\DeclareMathOperator{\lb}{\mathop{lb}}	% бинарный логарифм (логарифм по основанию 2)
\DeclareMathOperator{\supp}{\mathop{supp}}	% носитель

\renewcommand{\Im}{\mathop{\mathrm{Im}}\nolimits}	% мнимая часть
\renewcommand{\Re}{\mathop{\mathrm{Re}}\nolimits}	% вещественная часть

\renewcommand{\Prob}{\mathbb P}	% вероятность
\newcommand{\Expect}{\mathbb E}	% математическое ожидание
\renewcommand{\Variance}{\mathbb D}	% дисперсия
\newcommand{\Entropy}{\mathbb H}	% энтропия

%% Перенос знаков в формулах (по Львовскому)
\newcommand*{\hm}[1]{#1\nobreak\discretionary{}
	{\hbox{$\mathsurround=0pt #1$}}{}}

%%% Работа с картинками
\usepackage{graphicx}  % Для вставки рисунков
\graphicspath{{images/}{images2/}}  % папки с картинками
\setlength\fboxsep{3pt} % Отступ рамки \fbox{} от рисунка
\setlength\fboxrule{1pt} % Толщина линий рамки \fbox{}
\usepackage{wrapfig} % Обтекание рисунков и таблиц текстом
\RequirePackage{caption}
\DeclareCaptionLabelSeparator{defffis}{ -- }
\captionsetup{justification=centering,labelsep=defffis}

\renewcommand{\qedsymbol}{}

%%% Работа с таблицами
\usepackage{array,tabularx,tabulary,booktabs} % Дополнительная работа с таблицами
\usepackage{longtable}  % Длинные таблицы
\usepackage{multirow} % Слияние строк в таблице
\usepackage{xcolor}
\usepackage{colortbl}

\usepackage{tikz}
\usepackage{diagbox}
\usepackage{tikz}

\newtheorem{problem}{Задание}

\begin{document}
	
	\newcommand{\problemset}[1]{
		\begin{center}
			\Large #1
		\end{center}
	}
	
	\begin{tabbing}
	\hspace{11cm} \= Студент: \= Коротков Фёдор \\ % не забудьте исправить, студент Вы или студентка :)
																									% (а то некоторые забывают)
	\> Группа: \> 2362 \\  % Здесь меняете № группы
	\> Вариант: \> 34 \\    % А здесь меняете № варианта
	\> Дата: \> \today     % А вот здесь ничего не меняем!!!
\end{tabbing}
\hrule
\vspace{1cm}  % в данном файле меняем только Пол, Фамилию Имя, № группы и № варианта
	\problemset{Математическая логика и теория алгоритмов}
\problemset{Индивидуальное домашнее задание №2}

\renewcommand*{\proofname}{Решение}
\newcommand*\circled[1]{\tikz[baseline=(char.base)]{\node[shape=circle,draw,inner sep=2pt] (char) {#1};}}
\newcommand*\ccircled[1]{\tikz[baseline=(char.base)]
{\node[shape=circle,draw=black, fill=gray, opacity=0.8,inner sep=2pt] (char) {#1};}}
\newcommand*\bicircled[1]{\tikz[baseline=(char.base)]{\node[shape=circle,draw,inner sep=1pt] (char) {#1};}}
\shorthandoff{"}




\begin{problem}
Найдите резольвенты первого порядка следующего набора дизъюнктов $\overline{A} \vee \overline{B} \vee D \vee E, \overline{A} \vee B \vee C \vee D, B \vee C \vee \overline{E}, B \vee \overline{D}$
\end{problem}
\begin{proof} $ $\\
    $\alpha_1 := \overline{A} \vee \overline{B} \vee D \vee E$\\
    $\alpha_2 := \overline{A} \vee B \vee C \vee D$\\
    $\alpha_3 :=  B \vee C \vee \overline{E}$\\
    $\alpha_4 := B \vee \overline{D}$\\
    $Res(\alpha_1, \alpha_2) = \overline{A} \vee C \vee D \vee E$\\
    $Res(\alpha_1, \alpha_3) = 1$\\
    $Res(\alpha_1, \alpha_4) = 1$\\
    $Res(\alpha_2, \alpha_3) = \overline{A} \vee B \vee C \vee D \vee \overline{E}$\\
    $Res(\alpha_2, \alpha_4) = \overline{A} \vee B \vee C$\\
    $Res(\alpha_3, \alpha_4) = B \vee C \vee \overline{D} \vee \overline{E}$\\
\end{proof}


\begin{problem}
Приведите данную формулу к ПНФ:
$\overline{\overline{\forall xQ(x, x, x)} \equiv \forall y Q(y, y, c)}$
\end{problem}
\begin{proof} $ $\\
$\overline{\overline{\forall xQ(x, x, x)} \equiv \forall y Q(y, y, c)} = 
(\overline{\overline{\forall xQ(x, x, x)} \rightarrow \forall y Q(y, y, c)) \cdot (\forall y Q(y, y, c) \rightarrow \overline{\forall xQ(x, x, x)})} =$\\
$= (\overline{\forall xQ(x, x, x) \vee \forall y Q(y, y, c)) \cdot (\overline{\forall y Q(y, y, c)} \vee \overline{\forall xQ(x, x, x)})} =$\\
$= (\exists x \overline{Q(x, x, x)} \cdot \exists y\overline{Q(y ,y, c)}) \vee (\forall yQ(y ,y, c) \cdot \forall xQ(x, x, x)) =$\\
$= \exists x \exists y( \overline{Q(x, x, x)} \cdot \overline{Q(y ,y, c)}) \vee \forall y  \forall x(Q(y ,y, c) \cdot Q(x, x, x)) =$\\
$= \exists x \exists y \forall a \forall b [( \overline{Q(x, x, x)} \cdot \overline{Q(y ,y, c)}) \vee (Q(a, a, c) \cdot Q(b, b, b))] =$
\end{proof}


\begin{problem}
Приведите данную формулу к СНФ:\\
$\exists x \forall y \exists z \forall t \exists u ((R(t, u, x) \wedge Q(z)) \vee P(y))$
\end{problem}
\begin{proof} $ $\\
$\exists x \forall y \exists z \forall t \exists u ((R(t, u, x) \wedge Q(z)) \vee P(y)) = \exists x \forall y \exists z \forall t \exists u ((R(t, u, x) \vee P(y)) \wedge (Q(z) \vee P(y))) =$\\
$= \{x := c; z := f(y); u := g(y, t)\} = \forall y \forall t ((R(t, g(y, t), c) \vee P(y)) \wedge (Q(f(y)) \vee P(y)))$
\end{proof}


\begin{problem}
Унифицируйте данные формулы:\\
$Q(f(h(y)), h(r(g(y))), h(h(r(u))), q(q(c, a, b), t, g(y)), h(p(c)))$\\
$Q(f(h(b)), h(r(z)), h(h(r(\psi(a)))), q(q(c, a, b), s(x), z), h(x))$
\end{problem}
\begin{proof} $ $\\
$b = y$\\
$z = g(y)$\\
$u = \psi(a)$\\
$x = p(c) => s(x) = s(p(c))$\\
$t = s(p(c))$\\
Унификатор для данных предикатов:\\
$Q( f(h(y)), h(r(g(y))), h(h(r(\psi(a)))), q(q(c, a, b), s(p(c)), g(y)), h(p(c)) )$
\end{proof}


\begin{problem}
Придумайте интерпретацию, для которой данная формула а) верна; б) неверна; или докажите, что это невозможно:\\
$(Q(a, b, b) \oplus \exists y P(a, y, c)) \vee \forall x \overline{(R(x) \wedge S(c))}$
\end{problem}
\begin{proof} $ $\\

\end{proof}


\begin{problem}
а) Опишите язык, заданный данной грамматикой. б) Удовлетворяет ли он условию однозначности ветвления?\\
$A ::= s|sAA$
\end{problem}
\begin{proof} $ $\\
Язык, который задает такая грамматика представляет из себя $s_i...s_k; 1 \leq i \leq k; k$ - нечетное число\\
Язык не удовлетворяет условию однозначности по первому символу. Пусть $\alpha := s; \beta := sAA$.\\
Множества терминальных символов: $l(\alpha) = \{s\}; l(\beta) = \{s\}$\\
$l(\alpha) \cap l(\beta) \neq \emptyset \Rightarrow $ условие однозначности не выполняется
\end{proof}


\begin{problem}
Дана грамматика некоторого языка:\\
$S ::= C$\\
$B ::= aD$\\
$C ::= B$\\
$D ::= bC|bB|aD|aS|\wedge$\\
Постройте (любой) конечный автомат, распознающий этот
язык.
\end{problem}
\begin{proof} $ $\\
Упростим грамматику:\\
$S ::= aD$\\
$D ::= baD|aD|\wedge$\\
Все слова этого языка можно описать регулярным выражением $(a^+(ba)^*a^*)^+$\\
Конечный автомат, распознающий этот язык:\\
\begin{center}
\begin{tikzcd}[row sep=4em, column sep=4em]
\ccircled{S} \arrow[r, rightarrow, "a" description] & 
\bicircled{\circled{D}} \arrow[loop, in= 15, out = 75, looseness = 4,  "ba" description] \arrow[loop, in= 270, out = 350, looseness = 4,  "a" description]
\end{tikzcd}
\end{center}

\end{proof}


\begin{problem}
Постройте детерминированный конечный автомат, эквивалентный данному:\\
\begin{center}
\begin{tikzcd}[row sep=4em, column sep=4em]
\ccircled{q0} \ar[r, bend left = 10, rightarrow, "c" description] \ar[loop, in = 120, out = 180, looseness = 4, "a" description] \ar[dr, bend left = 10, rightarrow, "b" description] \ar[d, bend left = 10, rightarrow, "b" description]&
\circled{q2} \ar[loop, in = 15, out = 75, looseness = 4, "b" description] \ar[l, bend left = 10, rightarrow, "c" description]
\\
\circled{q1} \ar[u, bend left = 10, rightarrow, "c" description] \ar[ur, bend left = 10, rightarrow, "c" description] &
\bicircled{\circled{q3}} \ar[loop, out = 0, in = 300, looseness = 4, "b" description] \ar[l, bend left = 10, rightarrow, "\epsilon" description] \ar[ul, bend left = 10, rightarrow, "\epsilon" description] \ar[u, bend left = 10, rightarrow, "a" description]
\end{tikzcd}
\end{center}
\end{problem}
\begin{proof} $ $\\
Пустые переходы:\\
\begin{tikzcd}[row sep=4em, column sep=4em]
\circled{q1} & 
\bicircled{\circled{q3}} \ar[l, bend right = 10, rightarrow, "\epsilon" description] \ar[r, bend left = 10, rightarrow, "\epsilon" description] &
\circled{q0}
\end{tikzcd}\\
Итоговые состояния после исключения пустых переходов:\\
\begin{tikzcd}[row sep=4em, column sep=4em]
\circled{q1} &
\bicircled{\circled{q1q3}} &
\bicircled{\circled{q3q0}} &
\circled{q0}
\end{tikzcd}\\

Итоговый детерминированный конечный автомат:\\
\begin{center}
\begin{tikzcd}[row sep=4em, column sep=4em]
\circled{q0} \ar[loop, out = 180, in = 120, looseness = 4, "a" description] \ar[r, bend left = 10, rightarrow, "c" description] \ar[d, bend left = 10, rightarrow, "b" description] &
\circled{q2} \ar[loop, out = 90, in = 30, looseness = 4, "b" description] \ar[l, bend left = 10, rightarrow, "c" description]\\
\bicircled{\circled{q1q3}} \ar[loop, out = 270, in = 210, looseness = 4, "b" description] \ar[ur, bend left = 10, rightarrow, "a" description] \ar[r, bend left = 10, rightarrow, "c" description] &
\circled{q0q2} \ar[ul, bend left = 10, rightarrow, "a" description] \ar[ul, bend right = 10, rightarrow, "c" description] \ar[r, bend left = 10, rightarrow, "b" description]&
\bicircled{\circled{q1q2q3}} \ar[loop, out = 270, in = 210, looseness = 4, "b" description] \ar[ul, bend left = 10, rightarrow, "a" description] \ar[l, bend left = 10, rightarrow, "c" description]
\end{tikzcd}
\end{center}
\end{proof}


\begin{problem}
Постройте регулярное выражение, задающее язык, распознаваемый этим автоматом.\\
\begin{center}
\begin{tikzcd}[row sep=4em, column sep=4em]
\ccircled{q0} \ar[loop, out = 180, in = 120, looseness = 4, "c" description] \ar[d, bend left = 10, rightarrow, "b" description] &
\bicircled{\circled{q2}} \ar[loop, in = 30, out = 90, looseness = 4, "b" description] \ar[l, bend left = 10, rightarrow, "b" description]
\\
\circled{q1} \ar[loop, out = 270, in = 210, looseness = 4, "c" description] \ar[u, bend left = 10, rightarrow, "a" description] \ar[ur, bend left = 10, rightarrow, "c" description]
\end{tikzcd}
\end{center}
\end{problem}
\begin{proof} $ $\\
% Построим систему уравнений:\\
\begin{center}
\begin{equation*}
 \begin{cases}
   q_0 = cq_0 + bq_1
   \\
   q1 = cq_1 + aq_0 + cq_2
   \\
   q2 = bq_2 + bq_0 + \epsilon
 \end{cases}
\end{equation*}
$\Downarrow$
\begin{equation*}
 \begin{cases}
   q_0 = cq_0 + bq_1
   \\
   q1 = cq_1 + aq_0 + cb^*(bq_0+\epsilon) = cq_1 + (a+cb^*b)q_0 + cb^*)
   \\
   q2 = b^*(bq_0+\epsilon)
 \end{cases}
\end{equation*}
$\Downarrow$
\begin{equation*}
 \begin{cases}
   q_0 = cq_0+bc^*((a+cb^*b)q_0+cb^*)
   \\
   q1 = c^*((a+cb^*b)q_0+cb^*)
   \\
   q2 = b^*(bq_0+\epsilon)
 \end{cases}
\end{equation*}
$\Downarrow$
\begin{equation*}
 \begin{cases}
   q_0 = cq_0+bc^*aq_0 + bc^*cb^*bq_0+bc^*cb^*
   \\
   q1 = c^*((a+cb^*b)q_0+cb^*)
   \\
   q2 = b^*(bq_0+\epsilon)
 \end{cases}
\end{equation*}
\end{center}
Регулярное выражение, распознающее язык, задаваемый данным автоматом: $(c+bc^*a+bc^*cb^*b)^*bc^*cb^*$
\end{proof}


\begin{problem}
Постройте автомат, распознающий язык, задаваемый
этим регулярным выражением:\\
$ ((ab)^*((c+d)^*+e))^*$
\begin{center}
\end{center}
\end{problem}
\begin{proof} $ $\\
\begin{center}
\begin{tikzcd}[row sep=4em, column sep=4em]
\ccircled{S} \ar[r, bend left = 10, rightarrow, "a" description] &
\circled{q1} \ar[r, bend left = 10, rightarrow, "b" description] &
\circled{q2} \ar[ll, bend left = 20, rightarrow, "\epsilon" description] \ar[r, bend left = 10, rightarrow, "c" description] \ar[r, bend right = 10, rightarrow, "d" description] \ar[d, bend left = 10, rightarrow, "e" description]&
\bicircled{\circled{q3}} \ar[l, bend right = 30, rightarrow, "\epsilon" description] \ar[lll, bend left = 40, rightarrow, "\epsilon" description]
\\
&
&
\bicircled{\circled{q4}} \ar[ull, bend left = 20, rightarrow, "\epsilon" description]
\end{tikzcd}
\end{center}
\end{proof}


\begin{problem}
Постройте детерминированный конечный автомат, распознающий слова в алфавите \{c, p, z\} , которые оканчиваются на ppc.
\end{problem}
\begin{proof} $ $\\
\begin{center}
\begin{tikzcd}[row sep=4em, column sep=4em]
\ccircled{S} \ar[r, bend left = 30, rightarrow, "c" description] \ar[r, bend left = 10, rightarrow, "p" description] \ar[r, bend right = 10, rightarrow, "z" description] &
\circled{q1} \ar[l, bend left = 30, rightarrow, "\epsilon" description] \ar[r, bend left = 10, rightarrow, "p" description] &
\circled{q2} \ar[r, bend left = 10, rightarrow, "p" description] &
\circled{q3} \ar[r, bend left = 10, rightarrow, "c" description] &
\bicircled{\circled{q4}}
\end{tikzcd}
\end{center}
\end{proof}


\begin{problem}
Дана машина Тьюринга с начальным состоянием q0 и
конечным состоянием q2 . Какой результат даст эта машина Тьюринга для ленты otttt? Считывающая головка находится на крайнем левом символе. Пустой символ — *.\\
$1) q_1* \to q_1oR$\\
$2) q_1o \to q_2 * L$\\
$3) q_0* \to q_1oL$\\
$4) q_0o \to q_0*L$
\end{problem}
\begin{proof} $ $\\
$**\hat{o}tttt \xrightarrow{4}$
$*\hat{*}*tttt \xrightarrow{3}$
$\hat{*}o*tttt \xrightarrow{1}$
$o\hat{o}*tttt \xrightarrow{2}$
$\hat{o}**tttt$\\
Результатом работы машины Тьюринга будет o**tttt
\end{proof}


\begin{problem}
а) Что следующий алгоритм Маркова делает со словом
xzyxy?
б) Из некоторого слова после применения 1 шага данного алгоритма Маркова получилось yzyyzzy. Каким могло быть исходное слово?\\
$1) yx \to zyy$\\
$2) zz \to zy$\\
$3) xxx \to zy$\\
$4) zyz \to xx$\\
$5) yyy \to zzy$
\end{problem}
\begin{proof} $ $\\
a) $xzyxy \xrightarrow{1} xzzyyy \xrightarrow{2} xzyyyy \xrightarrow{5} xzzzyy \xrightarrow{2} xzyzyy \xrightarrow{4} xxxyy \xrightarrow{3} zyyy \xrightarrow{5} zzzy \xrightarrow{2} zyzy \xrightarrow{4} xxy$\\
б) $yyxzzy \xrightarrow{1} yzyyzzy; \quad yzzyzzy \xrightarrow{2} yzyyzzy; \quad yzyyzxxx \xrightarrow{3} yzyyzzy$
\end{proof}

\begin{problem}
Дан автомат с магазинной памятью. Входной алфавит
\{o, z\}; алфавит стека \{y\}; q0 — начальное состояние, q1 — конечное.
Правила\\
1) $q_0\epsilon o \to q_0y$\\
2) $q_0\epsilon z \to q_0\epsilon$\\
3) $q_0 yo \to q_1 \epsilon$\\
4) $q_0yz \to q_0 y$\\
5) $q_1\epsilon o \to q_0\epsilon$\\
6) $q_1\epsilon z \to q_0\epsilon$\\
7) $q_1 yo \to q_1y$\\
8) $q_1 yz \to q_0 y$\\
Придумайте пример шестибуквенного слова в алфавите
\{o, z\}, которое этот автомат а) распознаёт; б) не распознаёт.
в) Есть ли у этого автомата бесполезные правила, которые
не будут выполняться ни при каком поданном на вход автомата слове?
\end{problem}
\begin{proof} $ $\\
а) 
$
\hat{o}ozzoo
\begin{bmatrix}
    \quad
\end{bmatrix}
\xrightarrow{1}
o\hat{o}zzoo
\begin{bmatrix}
    y
\end{bmatrix}
\xrightarrow{3}
oo\hat{z}zoo
\begin{bmatrix}
    \quad
\end{bmatrix}
\xrightarrow{6}
ooz\hat{z}oo
\begin{bmatrix}
    \quad
\end{bmatrix}
\xrightarrow{2}
oozz\hat{o}o
\begin{bmatrix}
    \quad
\end{bmatrix}
\xrightarrow{1}
oozzo\hat{o}
\begin{bmatrix}
    y
\end{bmatrix}
\xrightarrow{3}
$
finish\\
б) 
$
\hat{o}ozzzz
\begin{bmatrix}
    \quad
\end{bmatrix}
\xrightarrow{1}
o\hat{o}zzzz
\begin{bmatrix}
    y
\end{bmatrix}
\xrightarrow{3}
oo\hat{z}zzz
\begin{bmatrix}
    \quad
\end{bmatrix}
\xrightarrow{6}
ooz\hat{z}zz
\begin{bmatrix}
    \quad
\end{bmatrix}
\xrightarrow{2}
oozz\hat{z}z
\begin{bmatrix}
    \quad
\end{bmatrix}
\xrightarrow{2}
oozzz\hat{z}
\begin{bmatrix}
    \quad
\end{bmatrix}
\xrightarrow{2}
q_0$ ($q_0$ не конечное состояние)\\
в) Никогда не будут выполнены правила 7 и 8, т.к. не существует правил, которые перевили бы автомат из состояния в состояние q1, чтобы при этом на верхушке стека оказался y.
\end{proof}
  % для удобства создаём по аналогии файлы ihw1.tex, ihw2.tex, etc
	                  % и просто меняем имя при компиляции
\end{document}