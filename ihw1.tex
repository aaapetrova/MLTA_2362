\problemset{Комбинаторика и теория графов}
\problemset{Индивидуальное домашнее задание №1}	% поменяйте номер ИДЗ

\renewcommand*{\proofname}{Решение}
%%%%%%%%%%%%%% ЗАДАНИЕ №1 %%%%%%%%%%%%%%
%% Условие задания №1
\begin{problem}
	Дано множество М = $\{96, 97, 33, 69, 72, 77, 19, 91\}$
    и следующие бинарные отношения на нем:
    \begin{itemize}
    
    \item $F_1(x,y) = 1 \Leftrightarrow \exists z \in M : (x - z)(y - z) < 0;$

    \item $F_2(x, y) = 1 \Leftrightarrow x \geq y$ поразрядно;

    \item $F_3(x, y) = 1 \Leftrightarrow [\frac{x}{5}] = [\frac{y}{5}]$;

    \item $F_4(x,y) = 1 \Leftrightarrow x^2 - y^3$ нечетно;

    \item $F_5(x, y) = 1 \Leftrightarrow |x-y| < 10$.
    \end{itemize}
    Для каждого из отношений:

    \begin{enumerate}

    \item[1.] Проверить, является ли бинарное отношение (далее -  б.о.) - рефлексивным, арефлексивным, симметричным, антисимметричным, асимметричным, транзитивным.
    \item[2.] Построить матрицы и графы этих б.о.

    \item[3.] Определить, являются ли эти б.о. отношениями эквивалентности, частичного порядка, линейного порядка, строгого порядка).

    \item[4.] Для отношений эквивалентности построить классы эквивалентности.

    \item[5.] Для отношений частичного порядка применить алгоритм топологической сортировки и получить отношение линейного порядка.

    \item[6.] Для нетранзитивных отношений построить транзитивное замыкание, используя алгоритм Уоршелла.
    \end{enumerate}
\end{problem}

%% Решение задания №1
\begin{proof}

\end{proof}