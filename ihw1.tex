\problemset{Комбинаторика и теория графов}
\problemset{Индивидуальное домашнее задание №1}	% поменяйте номер ИДЗ

\renewcommand*{\proofname}{Решение}

Дано множество \( M = \{34, 71, 42, 45, 52, 89, 28, 29\} \), отсортируем его для удобства: \[ M_s = \{28, 29, 34, 42, 45, 52, 71, 89\} \]

\begin{problem}
\[ F_1(x,y) = 1 \Leftrightarrow \exists z \in M : (x-z)(y-z)<0; \]
\[ (x - z)(y - z) < 0 \Leftrightarrow min(x, y) < z < max(x, y) \]

\end{problem}

\begin{proof} $ $
	\begin{itemize}
  	\item \textbf{Рефлексивность/арефлексивность \\}
  	\( \forall x \in M \)  \( (x - z)(x - z) = (x - z)^2 \geqslant 0 \) 
  	\( \forall z \in M \Rightarrow F_1(x,x) = 0 \) - отношение \textbf{арефлексивно}.
  	\item \textbf{Симметричность|асимметричность|асимметричность \\}
  	\( F_1(x, y) = 1 \Leftrightarrow 0 > (x - z)(y - z) = (y - z)(x - z) \Leftrightarrow F_1(y, x) = 1 \) - выполняется \textbf{симметричность}. \\
  	\( \exists x,y \in M, x \neq y : F_1(x,y) \land F_1(y, x) = 1 \) - \textbf{антисимметричность} не выполняется.
  	\item \textbf{Транзитивность \\}
  	\( \sqsupset
  	\begin{cases}
    \exists z_1 \in M : min(x,y) < z_1 < max(x,y) \Leftrightarrow F_1(x,y) = 1 \\
    \exists z_2 \in M : min(y,t) < z_2 < max(y,t) \Leftrightarrow F_1(y,t) = 1
  	\end{cases} \\ \)
  	Если $x$ и $t$ являются соседними элементами в $M_s$, то \( \nexists z_3 \in M : min(x,t) < z_3 < max(x,t) \Leftrightarrow F(x,t) = 0 \)  - \textbf{транзитивность} не выполняется. 
  	\item \textbf{Матрица смежности, граф. \\}
  	\bordermatrix{ & 28 & 29 & 34 & 42 & 45 & 52 & 71 & 89 \cr
28 & 0 & 0 & 1 & 1 & 1 & 1 & 1 & 1 \cr
29 & 0 & 0 & 0 & 1 & 1 & 1 & 1 & 1 \cr
34 & 1 & 0 & 0 & 0 & 1 & 1 & 1 & 1 \cr
42 & 1 & 1 & 0 & 0 & 0 & 1 & 1 & 1 \cr
45 & 1 & 1 & 1 & 0 & 0 & 0 & 1 & 1 \cr
52 & 1 & 1 & 1 & 1 & 0 & 0 & 0 & 1 \cr
71 & 1 & 1 & 1 & 1 & 1 & 0 & 0 & 0 \cr
89 & 1 & 1 & 1 & 1 & 1 & 1 & 0 & 0 } \\
  	
	\tikz {
		\path
(4.0, 8.0) node[state] (1) {28}
(6.83, 6.83) node[state] (2) {29}
(8.0, 4.0) node[state] (3) {34}
(6.83, 1.17) node[state] (4) {42}
(4.0, 0.0) node[state] (5) {45}
(1.17, 1.17) node[state] (6) {52}
(0.0, 4.0) node[state] (7) {71}
(1.17, 6.83) node[state] (8) {89};
\draw (1) -- (3);
\draw (1) -- (4);
\draw (1) -- (5);
\draw (1) -- (6);
\draw (1) -- (7);
\draw (1) -- (8);
\draw (2) -- (4);
\draw (2) -- (5);
\draw (2) -- (6);
\draw (2) -- (7);
\draw (2) -- (8);
\draw (3) -- (5);
\draw (3) -- (6);
\draw (3) -- (7);
\draw (3) -- (8);
\draw (4) -- (6);
\draw (4) -- (7);
\draw (4) -- (8);
\draw (5) -- (7);
\draw (5) -- (8);
\draw (6) -- (8);
	}
	\item Комбинация полученных свойств (\textbf{арефлексивность}, \textbf{симметричность}) не относится ни к одному отношению(эквивалентности, частичного порядка, линейного порядка, строгого порядка).
	\item Используя алгоритм Уоршелла построим транзитивное замыкание. \\
	\(
28 \rightarrow 42 \land 42 \rightarrow 29 \Rightarrow 28 \rightarrow 29 \\
29 \rightarrow 28 \land 28 \rightarrow 34 \Rightarrow 29 \rightarrow 34 \\
34 \rightarrow 28 \land 28 \rightarrow 42 \Rightarrow 34 \rightarrow 42 \\
42 \rightarrow 28 \land 28 \rightarrow 45 \Rightarrow 42 \rightarrow 45 \\
45 \rightarrow 28 \land 28 \rightarrow 52 \Rightarrow 45 \rightarrow 52 \\
52 \rightarrow 28 \land 28 \rightarrow 71 \Rightarrow 52 \rightarrow 71 \\
71 \rightarrow 28 \land 28 \rightarrow 89 \Rightarrow 71 \rightarrow 89 \)
	\\ \\ 
	\bordermatrix{ & 28 & 29 & 34 & 42 & 45 & 52 & 71 & 89  \cr
28 & 0 & 1 & 1 & 1 & 1 & 1 & 1 & 1 \cr
29 & 1 & 0 & 1 & 1 & 1 & 1 & 1 & 1 \cr
34 & 1 & 1 & 0 & 1 & 1 & 1 & 1 & 1 \cr
42 & 1 & 1 & 1 & 0 & 1 & 1 & 1 & 1 \cr
45 & 1 & 1 & 1 & 1 & 0 & 1 & 1 & 1 \cr
52 & 1 & 1 & 1 & 1 & 1 & 0 & 1 & 1 \cr
71 & 1 & 1 & 1 & 1 & 1 & 1 & 0 & 1 \cr
89 & 1 & 1 & 1 & 1 & 1 & 1 & 1 & 0 }
	\\ \\
	\tikz {
		\path
(4.0, 8.0) node[state] (1) {28}
(6.83, 6.83) node[state] (2) {29}
(8.0, 4.0) node[state] (3) {34}
(6.83, 1.17) node[state] (4) {42}
(4.0, 0.0) node[state] (5) {45}
(1.17, 1.17) node[state] (6) {52}
(0.0, 4.0) node[state] (7) {71}
(1.17, 6.83) node[state] (8) {89};
\draw (1) -- (3);
\draw (1) -- (4);
\draw (1) -- (5);
\draw (1) -- (6);
\draw (1) -- (7);
\draw (1) -- (8);
\draw (2) -- (4);
\draw (2) -- (5);
\draw (2) -- (6);
\draw (2) -- (7);
\draw (2) -- (8);
\draw (3) -- (5);
\draw (3) -- (6);
\draw (3) -- (7);
\draw (3) -- (8);
\draw (4) -- (6);
\draw (4) -- (7);
\draw (4) -- (8);
\draw (5) -- (7);
\draw (5) -- (8);
\draw (6) -- (8);
\draw[densely dotted] (1) -- (2);
\draw[densely dotted] (2) -- (3);
\draw[densely dotted] (3) -- (4);
\draw[densely dotted] (4) -- (5);
\draw[densely dotted] (5) -- (6);
\draw[densely dotted] (6) -- (7);
\draw[densely dotted] (7) -- (8);
	}
	\end{itemize}
\end{proof}

\begin{problem}
\[ F_2(x,y) = 1 \Leftrightarrow x \geqslant	 y \emph{ поразрядно}; \]
\end{problem}

\begin{proof} $ $
	\begin{itemize}
  	\item \textbf{Рефлексивность/арефлексивность \\}
  	\( \forall x \in M \) \( x = x \) поразрядно \( \Rightarrow x \geqslant x \) поразрядно \( \Leftrightarrow F_2(x, x) = 1 \) - отношение \textbf{рефлексивно}.
  	\item \textbf{Симметричность|асимметричность|асимметричность \\}
  	\( \sqsupset
  	\begin{cases}
    F_2(x,y) = 1 \Leftrightarrow x \geqslant y \\
    F_2(y,x) = 1 \Leftrightarrow y \geqslant x 
  	\end{cases} \Rightarrow x \geqslant y \leqslant x \Rightarrow y = x\) - отношение \textbf{антисимметрично}.
  	\item \textbf{Транзитивность \\}
  	\( \sqsupset
  	\begin{cases}
    F_2(x,y) = 1 \Leftrightarrow x \geqslant y \\
    F_2(y,z) = 1 \Leftrightarrow y \geqslant z
  	\end{cases} \Rightarrow x \geqslant y \geqslant z \Rightarrow x \geqslant z \Leftrightarrow F_2(x, z) = 1 \) - \textbf{транзитивность} выполняется.
  	\item \textbf{Матрица смежности, граф. \\}
  	\bordermatrix{ & 28 & 29 & 34 & 42 & 45 & 52 & 71 & 89 \cr
28 & 1 & 0 & 0 & 0 & 0 & 0 & 0 & 0 \cr
29 & 1 & 1 & 0 & 0 & 0 & 0 & 0 & 0 \cr
34 & 0 & 0 & 1 & 0 & 0 & 0 & 0 & 0 \cr
42 & 0 & 0 & 0 & 1 & 0 & 0 & 0 & 0 \cr
45 & 0 & 0 & 1 & 1 & 1 & 0 & 0 & 0 \cr
52 & 0 & 0 & 0 & 1 & 0 & 1 & 0 & 0 \cr
71 & 0 & 0 & 0 & 0 & 0 & 0 & 1 & 0 \cr
89 & 1 & 1 & 1 & 1 & 1 & 1 & 1 & 1 } \\

	\tikz {
		\path
(4.0, 8.0) node[state] (1) {28}
(6.83, 6.83) node[state] (2) {29}
(8.0, 4.0) node[state] (3) {34}
(6.83, 1.17) node[state] (4) {42}
(4.0, 0.0) node[state] (5) {45}
(1.17, 1.17) node[state] (6) {52}
(0.0, 4.0) node[state] (7) {71}
(1.17, 6.83) node[state] (8) {89};
\draw[->] (1) to [out=45.0,in=90.0,looseness=5] (1);
\draw[->] (2) -- (1);
\draw[->] (2) to [out=0.0,in=45.0,looseness=5] (2);
\draw[->] (3) to [out=-45.0,in=0.0,looseness=5] (3);
\draw[->] (4) to [out=-90.0,in=-45.0,looseness=5] (4);
\draw[->] (5) -- (3);
\draw[->] (5) -- (4);
\draw[->] (5) to [out=-135.0,in=-90.0,looseness=5] (5);
\draw[->] (6) -- (4);
\draw[->] (6) to [out=-180.0,in=-135.0,looseness=5] (6);
\draw[->] (7) to [out=-225.0,in=-180.0,looseness=5] (7);
\draw[->] (8) -- (1);
\draw[->] (8) -- (2);
\draw[->] (8) -- (3);
\draw[->] (8) -- (4);
\draw[->] (8) -- (5);
\draw[->] (8) -- (6);
\draw[->] (8) -- (7);
\draw[->] (8) to [out=-270.0,in=-225.0,looseness=5] (8);
	}
	\item Комбинация полученных свойств (\textbf{рефлексивность}, \textbf{антисимметричность}, \textbf{транзитивность}) относится к отношению \textbf{частичного порядка}. \\
	\( \exists x,y \in M : F_2(x,y) \lor F_2(y,x) = 0 \) - отношение не является отношением \textbf{линейного порядка}.	
	
	\item Применим алгоритм топологической сортировки и получим отношение линейного порядка. \\
	\\
	\bordermatrix{ & 28 & 29 & 34 & 42 & 45 & 52 & 71 & 89  \cr
28 & 1 & 0 & 0 & 0 & 0 & 0 & 0 & 0 \cr
29 & 1 & 1 & 0 & 0 & 0 & 0 & 0 & 0 \cr
34 & 1 & 1 & 1 & 0 & 0 & 0 & 0 & 0 \cr
42 & 1 & 1 & 1 & 1 & 0 & 0 & 0 & 0 \cr
45 & 1 & 1 & 1 & 1 & 1 & 0 & 0 & 0 \cr
52 & 1 & 1 & 1 & 1 & 1 & 1 & 0 & 0 \cr
71 & 1 & 1 & 1 & 1 & 1 & 1 & 1 & 0 \cr
89 & 1 & 1 & 1 & 1 & 1 & 1 & 1 & 1 }
	\\ \\
	\tikz {
		\path
(4.0, 8.0) node[state] (1) {$28_8$}
(6.83, 6.83) node[state] (2) {$29_7$}
(8.0, 4.0) node[state] (3) {$34_6$}
(6.83, 1.17) node[state] (4) {$42_5$}
(4.0, 0.0) node[state] (5) {$45_4$}
(1.17, 1.17) node[state] (6) {$52_3$}
(0.0, 4.0) node[state] (7) {$71_2$}
(1.17, 6.83) node[state] (8) {$89_1$};
\draw[->] (1) to [out=45.0,in=90.0,looseness=5] (1);
\draw[->] (2) -- (1);
\draw[->] (2) to [out=0.0,in=45.0,looseness=5] (2);
\draw[->] (3) to [out=-45.0,in=0.0,looseness=5] (3);
\draw[->] (4) to [out=-90.0,in=-45.0,looseness=5] (4);
\draw[->] (5) -- (3);
\draw[->] (5) -- (4);
\draw[->] (5) to [out=-135.0,in=-90.0,looseness=5] (5);
\draw[->] (6) -- (4);
\draw[->] (6) to [out=-180.0,in=-135.0,looseness=5] (6);
\draw[->] (7) to [out=-225.0,in=-180.0,looseness=5] (7);
\draw[->] (8) -- (1);
\draw[->] (8) -- (2);
\draw[->] (8) -- (3);
\draw[->] (8) -- (4);
\draw[->] (8) -- (5);
\draw[->] (8) -- (6);
\draw[->] (8) -- (7);
\draw[->] (8) to [out=-270.0,in=-225.0,looseness=5] (8);
\draw[->][densely dotted] (7) -- (1);
\draw[->][densely dotted] (7) -- (2);
\draw[->][densely dotted] (7) -- (3);
\draw[->][densely dotted] (7) -- (4);
\draw[->][densely dotted] (7) -- (5);
\draw[->][densely dotted] (7) -- (6);
\draw[->][densely dotted] (6) -- (1);
\draw[->][densely dotted] (6) -- (2);
\draw[->][densely dotted] (6) -- (3);
\draw[->][densely dotted] (6) -- (5);
\draw[->][densely dotted] (5) -- (1);
\draw[->][densely dotted] (5) -- (2);
\draw[->][densely dotted] (4) -- (1);
\draw[->][densely dotted] (4) -- (2);
\draw[->][densely dotted] (4) -- (3);
\draw[->][densely dotted] (3) -- (1);
\draw[->][densely dotted] (3) -- (2);
	}	
	
  	\end{itemize}
\end{proof}

\begin{problem}
\[ F_3(x,y) = 1 \Leftrightarrow \left[ \frac{x}{5} \right] = \left[ \frac{y}{5} \right]; \]
\end{problem}

\begin{proof} $ $
	\begin{itemize}
  	\item \textbf{Рефлексивность/арефлексивность \\}
  	\( \forall x \in M \) \( \left[ \frac{x}{5} \right] = \left[ \frac{x}{5} \right] \Leftrightarrow F_3(x,x) = 1 \) - отношение \textbf{рефлексивно}.
  	\item \textbf{Симметричность|асимметричность|асимметричность \\}
  	\( F_3(x,y) = 1 \Leftrightarrow \left[ \frac{x}{5} \right] = \left[ \frac{y}{5} \right] \Leftrightarrow \left[ \frac{y}{5} \right] = \left[ \frac{x}{5} \right] \Leftrightarrow F_3(y,x) = 1 \) - отношение \textbf{симметрично}. \\
  	\( \exists x,y \in M, x \neq y : F_3(x,y) \land F_3(y, x) = 1 \) - \textbf{антисимметричность} не выполняется.
  	\item \textbf{Транзитивность \\}
  	\( \sqsupset
  	\begin{cases}
    F_3(x,y) = 1 \Leftrightarrow \left[ \frac{x}{5} \right] = \left[ \frac{y}{5} \right] \\
    F_3(y,z) = 1 \Leftrightarrow \left[ \frac{y}{5} \right] = \left[ \frac{z}{5} \right]   
  	\end{cases} \Rightarrow \left[ \frac{x}{5} \right] = \left[ \frac{z}{5} \right] \Leftrightarrow F_3(x,z) = 1 \) - \textbf{транзитивность} выполняется.
  	\item \textbf{Матрица смежности, граф. \\}
  	\bordermatrix{ & 28 & 29 & 34 & 42 & 45 & 52 & 71 & 89  \cr
28 & 1 & 1 & 0 & 0 & 0 & 0 & 0 & 0 \cr
29 & 1 & 1 & 0 & 0 & 0 & 0 & 0 & 0 \cr
34 & 0 & 0 & 1 & 0 & 0 & 0 & 0 & 0 \cr
42 & 0 & 0 & 0 & 1 & 0 & 0 & 0 & 0 \cr
45 & 0 & 0 & 0 & 0 & 1 & 0 & 0 & 0 \cr
52 & 0 & 0 & 0 & 0 & 0 & 1 & 0 & 0 \cr
71 & 0 & 0 & 0 & 0 & 0 & 0 & 1 & 0 \cr
89 & 0 & 0 & 0 & 0 & 0 & 0 & 0 & 1 }
	\\ \\ 
	\tikz {
		\path
(4.0, 8.0) node[state] (1) {28}
(6.83, 6.83) node[state] (2) {29}
(8.0, 4.0) node[state] (3) {34}
(6.83, 1.17) node[state] (4) {42}
(4.0, 0.0) node[state] (5) {45}
(1.17, 1.17) node[state] (6) {52}
(0.0, 4.0) node[state] (7) {71}
(1.17, 6.83) node[state] (8) {89};
\draw[->] (1) to [out=45.0,in=90.0,looseness=5] (1);
\draw (1) -- (2);
\draw[->] (2) to [out=0.0,in=45.0,looseness=5] (2);
\draw[->] (3) to [out=-45.0,in=0.0,looseness=5] (3);
\draw[->] (4) to [out=-90.0,in=-45.0,looseness=5] (4);
\draw[->] (5) to [out=-135.0,in=-90.0,looseness=5] (5);
\draw[->] (6) to [out=-180.0,in=-135.0,looseness=5] (6);
\draw[->] (7) to [out=-225.0,in=-180.0,looseness=5] (7);
\draw[->] (8) to [out=-270.0,in=-225.0,looseness=5] (8);
}
	\item Комбинация полученных свойств (\textbf{рефлексивность}, \textbf{симметричность}, \textbf{транзитивность}) относится к отношению \textbf{эквивалентности}. \\
	\item Построим классы эквивалентности. \\
	На графе каждому классу эквивалентности соответствует компонента связности. \\
	\{28, 29\}, \{34\}, \{42\}, \{45\}, \{52\}, \{71\}, \{89\}
  	\end{itemize} 

\end{proof}

\begin{problem}
\[ F_4(x,y) = 1 \Leftrightarrow x^2-y^3	\emph{ четно}; \]
\[ x^2 - y^3 \equiv 0 ~(mod~2) \Leftrightarrow x^2 \equiv y^3 ~(mod~2) \Leftrightarrow x \equiv y ~(mod~2) \]
\end{problem}
	
\begin{proof} $ $
	\begin{itemize}
  	\item \textbf{Рефлексивность/арефлексивность \\}
  	\( \forall x \in M ~ x \equiv x ~(mod~2) \Leftrightarrow F_4(x,x) = 1 \) - \textbf{рефлексивность} выполняется.
  	\item \textbf{Симметричность|асимметричность|асимметричность \\}
  	\( F_4(x,y) = 1 \Leftrightarrow x \equiv y ~(mod~2) \Leftrightarrow y \equiv x ~(mod~2) \Leftrightarrow F_4(y,x) = 1 \) - отношение \textbf{симметрично}.
  	\( \exists x,y \in M, x \neq y : F_4(x,y) \land F_4(y, x) = 1 \) - \textbf{антисимметричность} не выполняется.
  	\item \textbf{Транзитивность \\}
  	\( \sqsupset
  	\begin{cases}
    F_4(x,y) = 1 \Leftrightarrow x \equiv y ~(mod~2) \\
    F_4(y,z) = 1 \Leftrightarrow y \equiv z ~(mod~2)   
  	\end{cases} \Rightarrow x \equiv z ~(mod~2) \Leftrightarrow F_4(x,z) = 1 \) - \textbf{транзитивность} выполняется.
  	\item \textbf{Матрица смежности, граф. \\}
  	\bordermatrix{ & 28 & 29 & 34 & 42 & 45 & 52 & 71 & 89  \cr
28 & 1 & 0 & 1 & 1 & 0 & 1 & 0 & 0 \cr
29 & 0 & 1 & 0 & 0 & 1 & 0 & 1 & 1 \cr
34 & 1 & 0 & 1 & 1 & 0 & 1 & 0 & 0 \cr
42 & 1 & 0 & 1 & 1 & 0 & 1 & 0 & 0 \cr
45 & 0 & 1 & 0 & 0 & 1 & 0 & 1 & 1 \cr
52 & 1 & 0 & 1 & 1 & 0 & 1 & 0 & 0 \cr
71 & 0 & 1 & 0 & 0 & 1 & 0 & 1 & 1 \cr
89 & 0 & 1 & 0 & 0 & 1 & 0 & 1 & 1 }
	\\ \\
	\tikz {
		\path
(4.0, 8.0) node[state] (1) {28}
(6.83, 6.83) node[state] (2) {29}
(8.0, 4.0) node[state] (3) {34}
(6.83, 1.17) node[state] (4) {42}
(4.0, 0.0) node[state] (5) {45}
(1.17, 1.17) node[state] (6) {52}
(0.0, 4.0) node[state] (7) {71}
(1.17, 6.83) node[state] (8) {89};
\draw[->][dashed] (1) to [out=45.0,in=90.0,looseness=5] (1);
\draw[dashed] (1) -- (3);
\draw[dashed] (1) -- (4);
\draw[dashed] (1) -- (6);
\draw[->] (2) to [out=0.0,in=45.0,looseness=5] (2);
\draw (2) -- (5);
\draw (2) -- (7);
\draw (2) -- (8);
\draw[->][dashed] (3) to [out=-45.0,in=0.0,looseness=5] (3);
\draw[dashed] (3) -- (4);
\draw[dashed] (3) -- (6);
\draw[->][dashed] (4) to [out=-90.0,in=-45.0,looseness=5] (4);
\draw[dashed] (4) -- (6);
\draw[->] (5) to [out=-135.0,in=-90.0,looseness=5] (5);
\draw (5) -- (7);
\draw (5) -- (8);
\draw[->][dashed] (6) to [out=-180.0,in=-135.0,looseness=5] (6);
\draw[->] (7) to [out=-225.0,in=-180.0,looseness=5] (7);
\draw (7) -- (8);
\draw[->] (8) to [out=-270.0,in=-225.0,looseness=5] (8);
}
	\item Комбинация полученных свойств (\textbf{рефлексивность}, \textbf{симметричность}, \textbf{транзитивность}) относится к отношению \textbf{эквивалентности}. \\
	\item Построим классы эквивалентности. \\
	На графе каждому классу эквивалентности соответствует компонента связности. \\
	\{28, 34, 42, 52\}, \{29, 45, 71, 89\}
  	\end{itemize}
\end{proof}

\begin{problem}
\[ F_5(x,y) = 1 \Leftrightarrow |x-y|<10. \]
\end{problem}

\begin{proof} $ $
	\begin{itemize}
  	\item \textbf{Рефлексивность/арефлексивность \\}
  	\( \forall x \in M ~ |x - x| = 0 < 10 \Leftrightarrow F_5(x,x) = 1 \) - отношение \textbf{рефлексивно}.
  	\item \textbf{Симметричность|асимметричность|асимметричность \\}
  	\( F_5(x,y) = 1 \Leftrightarrow 10>|x-y|=|-(y-x)|=|y-x| \Leftrightarrow F_5(y,x) = 1 \) - отношение \textbf{симметрично}.
  	\( \exists x,y \in M, x \neq y : F_5(x,y) \land F_5(y, x) = 1 \) - \textbf{антисимметричность} не выполняется.
  	\item \textbf{Транзитивность \\}
  	\( \sqsupset
  	\begin{cases}
    F_5(x,y) = 1 \Leftrightarrow |x-y|<10 \\
    F_5(y,z) = 1 \Leftrightarrow |y-z|<10 \\
    |x-z|>=10 \Leftrightarrow F_5(x,z) = 0   
  	\end{cases} \) - \textbf{транзитивность} не выполняется. \\
  	Контрпример: \\
  	\( \begin{cases}
    F_5(34,42) = 1 \\
    F_5(42,45) = 1 \\
    F_5(34,45) = 0  
  	\end{cases}  \)
  	\item \textbf{Матрица смежности, граф. \\}
  	\bordermatrix{ & 28 & 29 & 34 & 42 & 45 & 52 & 71 & 89  \cr
28 & 1 & 1 & 1 & 0 & 0 & 0 & 0 & 0 \cr
29 & 1 & 1 & 1 & 0 & 0 & 0 & 0 & 0 \cr
34 & 1 & 1 & 1 & 1 & 0 & 0 & 0 & 0 \cr
42 & 0 & 0 & 1 & 1 & 1 & 0 & 0 & 0 \cr
45 & 0 & 0 & 0 & 1 & 1 & 1 & 0 & 0 \cr
52 & 0 & 0 & 0 & 0 & 1 & 1 & 0 & 0 \cr
71 & 0 & 0 & 0 & 0 & 0 & 0 & 1 & 0 \cr
89 & 0 & 0 & 0 & 0 & 0 & 0 & 0 & 1 }
	\\ \\ 
	\tikz {
		\path
(4.0, 8.0) node[state] (1) {28}
(6.83, 6.83) node[state] (2) {29}
(8.0, 4.0) node[state] (3) {34}
(6.83, 1.17) node[state] (4) {42}
(4.0, 0.0) node[state] (5) {45}
(1.17, 1.17) node[state] (6) {52}
(0.0, 4.0) node[state] (7) {71}
(1.17, 6.83) node[state] (8) {89};
\draw[->] (1) to [out=45.0,in=90.0,looseness=5] (1);
\draw (1) -- (2);
\draw (1) -- (3);
\draw[->] (2) to [out=0.0,in=45.0,looseness=5] (2);
\draw (2) -- (3);
\draw[->] (3) to [out=-45.0,in=0.0,looseness=5] (3);
\draw (3) -- (4);
\draw[->] (4) to [out=-90.0,in=-45.0,looseness=5] (4);
\draw (4) -- (5);
\draw[->] (5) to [out=-135.0,in=-90.0,looseness=5] (5);
\draw (5) -- (6);
\draw[->] (6) to [out=-180.0,in=-135.0,looseness=5] (6);
\draw[->] (7) to [out=-225.0,in=-180.0,looseness=5] (7);
\draw[->] (8) to [out=-270.0,in=-225.0,looseness=5] (8);
}
	\item Комбинация полученных свойств (\textbf{рефлексивность}, \textbf{симметричность}) не относится ни к одному отношению(эквивалентности, частичного порядка, линейного порядка, строгого порядка).
	\item Используя алгоритм Уоршелла построим транзитивное замыкание. \\
	\( 28 \rightarrow 34 \land 34 \rightarrow 42 \Rightarrow 28 \rightarrow 42 \\
28 \rightarrow 42 \land 42 \rightarrow 45 \Rightarrow 28 \rightarrow 45 \\
28 \rightarrow 45 \land 45 \rightarrow 52 \Rightarrow 28 \rightarrow 52 \\
29 \rightarrow 28 \land 28 \rightarrow 42 \Rightarrow 29 \rightarrow 42 \\
29 \rightarrow 28 \land 28 \rightarrow 45 \Rightarrow 29 \rightarrow 45 \\
29 \rightarrow 28 \land 28 \rightarrow 52 \Rightarrow 29 \rightarrow 52 \\
34 \rightarrow 28 \land 28 \rightarrow 45 \Rightarrow 34 \rightarrow 45 \\
34 \rightarrow 28 \land 28 \rightarrow 52 \Rightarrow 34 \rightarrow 52 \\
42 \rightarrow 28 \land 28 \rightarrow 52 \Rightarrow 42 \rightarrow 52 \)
	\\ \\ 
	\tikz {
		\path
(4.0, 8.0) node[state] (1) {28}
(6.83, 6.83) node[state] (2) {29}
(8.0, 4.0) node[state] (3) {34}
(6.83, 1.17) node[state] (4) {42}
(4.0, 0.0) node[state] (5) {45}
(1.17, 1.17) node[state] (6) {52}
(0.0, 4.0) node[state] (7) {71}
(1.17, 6.83) node[state] (8) {89};
\draw[->] (1) to [out=45.0,in=90.0,looseness=5] (1);
\draw (1) -- (2);
\draw (1) -- (3);
\draw[->] (2) to [out=0.0,in=45.0,looseness=5] (2);
\draw (2) -- (3);
\draw[->] (3) to [out=-45.0,in=0.0,looseness=5] (3);
\draw (3) -- (4);
\draw[->] (4) to [out=-90.0,in=-45.0,looseness=5] (4);
\draw (4) -- (5);
\draw[->] (5) to [out=-135.0,in=-90.0,looseness=5] (5);
\draw (5) -- (6);
\draw[->] (6) to [out=-180.0,in=-135.0,looseness=5] (6);
\draw[->] (7) to [out=-225.0,in=-180.0,looseness=5] (7);
\draw[->] (8) to [out=-270.0,in=-225.0,looseness=5] (8);
\draw[densely dotted] (1) -- (4);
\draw[densely dotted] (1) -- (5);
\draw[densely dotted] (1) -- (6);
\draw[densely dotted] (2) -- (4);
\draw[densely dotted] (2) -- (5);
\draw[densely dotted] (2) -- (6);
\draw[densely dotted] (3) -- (5);
\draw[densely dotted] (3) -- (6);
\draw[densely dotted] (4) -- (6);
}
  	\end{itemize}
\end{proof}